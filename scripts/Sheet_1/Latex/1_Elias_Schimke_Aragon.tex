\documentclass{article}
\usepackage[utf8]{inputenc}
\usepackage{graphicx}
\usepackage{amsmath}
\usepackage{hyperref}

\title{Problem Sheet 1}
\author{Robin Elias \and Piet Schimke \and Andrea Farfan Aragon}
\date{}

\begin{document}

\maketitle
\begin{center}
    All the programming task are solved using the julia programming language.
    To reproduce the solution pull \href{https://github.com/relias96/Epidemiologie}{https://github.com/relias96/Epidemiologie} and following the Instructions in the README\.md file.
    Then run scripts\Sheet1\T1.jl
\end{center}
 
\section*{A}
Given the DGL $\frac{dI}{dT}=\beta SI- \gamma I $ and the assumption $S = N = const.$ can be simplified:

$$\frac{dI}{dT}=\beta NI- \gamma I = I(\beta N- \gamma)$$

Since this DGL is linear in I, the solution of the DGL can be determined as:

$$ I(t) = I_0 *e^{\beta N - \gamma}$$

\section*{B}



\begin{figure}[htbp]
    \centering
    \includegraphics[width = 0.8\linewidth]{../../../plots/T1/Germany.png}
    \caption{Germany}
    \label{fig:Germany}
\end{figure}

\begin{figure}[htbp]
    \centering
    \includegraphics[width = 0.8\linewidth]{../../../plots/T1/Italy.png}
    \caption{Italy}
    \label{fig:Italy}
\end{figure}

\begin{figure}[htbp]
    \centering
    \includegraphics[width = 0.8\linewidth]{../../../plots/T1/China.png}
    \caption{China}
    \label{fig:China}
\end{figure}


\begin{tabular}{|c|c|c|}
    \hline
    Country & $\lambda$ &$R_0$  \\
    \hline
    Germany &0.203     &2.62   \\
    Italy	 &0.199     &2.59   \\
    China	 &0.028     &1.224  \\
    \hline
\end{tabular}



\section*{C}

Our Model only fits under the disease-free equilibrium assumtion. For longer timeseries this assumtion is violated and the Model no longer fits the data as well as for short timeseries.

\section*{D}

Since $R_0 = \frac{\beta N}{\gamma}$ is density dependent in $N$ China should have a higher $R_0$ Value. However, because our model cannot account for policies such as quarantine, our model is inaccurate in that respect.



\end{document}
